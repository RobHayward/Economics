\documentclass[12pt, a4paper, oneside]{article}\usepackage[]{graphicx}\usepackage[]{color}
%% maxwidth is the original width if it is less than linewidth
%% otherwise use linewidth (to make sure the graphics do not exceed the margin)
\makeatletter
\def\maxwidth{ %
  \ifdim\Gin@nat@width>\linewidth
    \linewidth
  \else
    \Gin@nat@width
  \fi
}
\makeatother

\definecolor{fgcolor}{rgb}{0.345, 0.345, 0.345}
\newcommand{\hlnum}[1]{\textcolor[rgb]{0.686,0.059,0.569}{#1}}%
\newcommand{\hlstr}[1]{\textcolor[rgb]{0.192,0.494,0.8}{#1}}%
\newcommand{\hlcom}[1]{\textcolor[rgb]{0.678,0.584,0.686}{\textit{#1}}}%
\newcommand{\hlopt}[1]{\textcolor[rgb]{0,0,0}{#1}}%
\newcommand{\hlstd}[1]{\textcolor[rgb]{0.345,0.345,0.345}{#1}}%
\newcommand{\hlkwa}[1]{\textcolor[rgb]{0.161,0.373,0.58}{\textbf{#1}}}%
\newcommand{\hlkwb}[1]{\textcolor[rgb]{0.69,0.353,0.396}{#1}}%
\newcommand{\hlkwc}[1]{\textcolor[rgb]{0.333,0.667,0.333}{#1}}%
\newcommand{\hlkwd}[1]{\textcolor[rgb]{0.737,0.353,0.396}{\textbf{#1}}}%

\usepackage{framed}
\makeatletter
\newenvironment{kframe}{%
 \def\at@end@of@kframe{}%
 \ifinner\ifhmode%
  \def\at@end@of@kframe{\end{minipage}}%
  \begin{minipage}{\columnwidth}%
 \fi\fi%
 \def\FrameCommand##1{\hskip\@totalleftmargin \hskip-\fboxsep
 \colorbox{shadecolor}{##1}\hskip-\fboxsep
     % There is no \\@totalrightmargin, so:
     \hskip-\linewidth \hskip-\@totalleftmargin \hskip\columnwidth}%
 \MakeFramed {\advance\hsize-\width
   \@totalleftmargin\z@ \linewidth\hsize
   \@setminipage}}%
 {\par\unskip\endMakeFramed%
 \at@end@of@kframe}
\makeatother

\definecolor{shadecolor}{rgb}{.97, .97, .97}
\definecolor{messagecolor}{rgb}{0, 0, 0}
\definecolor{warningcolor}{rgb}{1, 0, 1}
\definecolor{errorcolor}{rgb}{1, 0, 0}
\newenvironment{knitrout}{}{} % an empty environment to be redefined in TeX

\usepackage{alltt} % Paper size, default font size and one-sided paper
%\graphicspath{{./Figures/}} % Specifies the directory where pictures are stored
%\usepackage[dcucite]{harvard}
\usepackage{amsmath}
\usepackage{setspace}
\usepackage{pdflscape}
\usepackage{rotating}
\usepackage[flushleft]{threeparttable}
\usepackage{multirow}
\usepackage[comma, sort&compress]{natbib}% Use the natbib reference package - read up on this to edit the reference style; if you want text (e.g. Smith et al., 2012) for the in-text references (instead of numbers), remove 'numbers' 
\usepackage{graphicx}
%\bibliographystyle{plainnat}
\bibliographystyle{agsm}
\usepackage[colorlinks = true, citecolor = blue, linkcolor = blue]{hyperref}
%\hypersetup{urlcolor=blue, colorlinks=true} % Colors hyperlinks in blue - change to black if annoying
%\renewcommand[\harvardurl]{URL: \url}
\IfFileExists{upquote.sty}{\usepackage{upquote}}{}
\begin{document}
\title{Economics Ideas}
%\author{Rob Hayward\footnote{University of Brighton Business School, Lewes Road, Brighton, BN2 4AT; Telephone 01273 642586.  rh49@brighton.ac.uk}}
\date{\today}
\maketitle

\section*{PiKetty:Capital}
Piketty:  \emph{Capital in the Twenty-First Century}.  The focus is rising inequality.  The rise in inequality is about the top 0.1\%.  The essential point of Piketty is that inherited wealth is making a comeback.  \href{http://www.ft.com/cms/s/2/0c6e9302-c3e2-11e3-a8e0-00144feabdc0.html#axzz2z7YbuMqh}{Martin Woolf} summarises Piketty's book as:
\begin{itemize}
\item No tendency towards economic equality
\item Fall in inequality after 1945 was due to deliberate policy and the destruction of established establshed wealth between 1914 and 1945
\item nonhuman capital appears to be indespensible today as ever - disabusing the idea that human capital will become more important.
\item Wealth to income in Europe has risen above US levels in Europe (particularly France and UK)
\item The rise of the \emph{supermanager} in the US and the return of \emph{patrimonial capitalism} in the US.
\item Increase in US top earnings is explained by managers not sports or entertainment stars
\item The case against increased managerial earnings being the result of marginal product is made by the lack of evidence in any underlying economic improvement since the 1960s.  
\item It appears to be more about lower tax and changes in social norms.
\end{itemize}

At the heart is a view of capital accumulation. 
\begin{itemize}
\item The ratio of capital to income will rise without limit so long as the rate of return is higher than economic growth
\item This is usually the case (outside of destruction or appropriation of wealth or the burst of economic growth in post-war Europe or today's emerging economies)
\item Two reasons: 
\begin{itemize}
\item The rate of retun is only modestly related to the ratio of capital-to-income.  The \emph{elasticity of substitutio} between capital and labour is greater than one.  
\item In normal times, capitalists save a sufficient proprtion of their income to ensure that captial grows faster than the economy.  The most wealthy enjoy the highest returns.
\end{itemize}
\end{itemize}


\end{document}
